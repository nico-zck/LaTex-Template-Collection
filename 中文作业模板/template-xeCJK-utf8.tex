%!!!!!注意:该模板只能用XeLaTex编译!!!!!
\documentclass[UTF8,12pt]{article} % 12pt 为字号大小
\usepackage{amssymb,amsfonts,amsmath,amsthm}
%\usepackage{fontspec,xltxtra,xunicode}
%\usepackage{times}

%==============定义中文环境和字体================
\usepackage{xeCJK}
%定义mac和win的字体,两者选择一个即可

%定义mac字体
\setCJKmainfont[BoldFont={Heiti SC Light},ItalicFont={Kaiti SC Regular}]{Songti SC Regular}
\setCJKsansfont{Heiti SC Light}
\setCJKfamilyfont{zhsong}{Songti SC Regular}
\setCJKfamilyfont{zhhei}{Heiti SC Light}
\setCJKfamilyfont{zhkai}{Kaiti SC Regular}
\setCJKfamilyfont{zhfs}{STFangsong}
\setCJKfamilyfont{zhli}{Libian SC Regular}
\setCJKfamilyfont{zhyou}{Yuanti SC Regular}

%%定义windows字体
%\setCJKmainfont[BoldFont={SimHei},ItalicFont={KaiTi}]{SimSun}
%\setCJKsansfont{SimHei}
%\setCJKfamilyfont{zhsong}{SimSun}
%\setCJKfamilyfont{zhhei}{SimHei}
%\setCJKfamilyfont{zhkai}{KaiTi}
%\setCJKfamilyfont{zhfs}{FangSong}
%\setCJKfamilyfont{zhli}{LiSu}
%\setCJKfamilyfont{zhyou}{YouYuan}

\newcommand*{\songti}{\CJKfamily{zhsong}} 
\newcommand*{\heiti}{\CJKfamily{zhhei}}   
\newcommand*{\kaiti}{\CJKfamily{zhkai}}  
\newcommand*{\fangsong}{\CJKfamily{zhfs}} % 仿宋
\newcommand*{\lishu}{\CJKfamily{zhli}}    % 隶书
\newcommand*{\yuanti}{\CJKfamily{zhyou}} % 圆体
%==============================================

%----------
% 版面设置
%----------
%首段缩进
\usepackage{indentfirst}
\setlength{\parindent}{2em}
%行距
\usepackage{setspace}   %使用setspace可单独控制行距
\renewcommand{\baselinestretch}{1.5}    %定义主行距
%页边距
\usepackage[a4paper]{geometry}
\geometry{verbose,
  tmargin=2cm,% 上边距
  bmargin=2cm,% 下边距
  lmargin=3cm,% 左边距
  rmargin=3cm % 右边距
}

%----------
% 其他宏包
%----------
%图形相关
\usepackage[x11names]{xcolor} % must before tikz, x11names defines RoyalBlue3
\usepackage{graphicx}
\usepackage{pstricks,pst-plot,pst-eps}
\usepackage{subfig}
\def\pgfsysdriver{pgfsys-dvipdfmx.def} % put before tikz
\usepackage{tikz}

\usepackage{verbatim} %原文照排
\usepackage{enumerate}  %使用enumerate控制标题格式
\usepackage{url} %网址

%----------
% 习题与解答环境,用不到可以去掉
%----------
%习题环境
\theoremstyle{definition} 
\newtheorem{exs}{习题}

%解答环境
\ifx\proof\undefined
\newenvironment{proof}[1][\protect\proofname]
{\par
\normalfont\topsep6\p@\@plus6\p@\relax
\trivlist
\itemindent\parindent
\item[\hskip\labelsep\scshape #1]\ignorespaces
}{\endtrivlist\@endpefalse}
\providecommand{\proofname}{Proof}
\fi
%将Proof更改为中文
\renewcommand{\proofname}{\it{证明}}

%==========
% 正文部分
%==========

\begin{document}

\title{中文模板}
\author{Nico}
%\date{June 10, 2018} % 去掉前面注释即可取消自动日期;{ } 中可以自定义日期格式
\maketitle


%单独控制行距和字体
\section*{模板介绍}
\begin{spacing}{1.0}
{\kaiti 这个模板是UTF-8编码的,使用xeCJK宏包,中英文混排更美观,但编译速度稍慢。注意:该模板只能用xelatex编译。这个模板是UTF-8编码的,使用xeCJK宏包,中英文混排更美观,但编译速度稍慢。注意:该模板只能用xelatex编译。}
\end{spacing}


\section{中文字体}

\subsection{字体切换}
默认字体为宋体。我们可以这样来改变中文字体:
{\heiti 从现在起是黑体。}
{\kaiti 从现在起是楷体。}
{\fangsong 从现在起是仿宋。}
{\lishu 从现在起是隶书。}
{\yuanti 从现在起是圆体。}
从现在起又是宋体。

\subsection{字体强调}
加粗字体自动变为黑体:\textbf{粗体},加斜或强调字体自动变成楷体:\textit{斜体},\emph{强调}。


\section{例子}

\subsection{标题1}
无序号
\begin{itemize}
    \item 这是第一项
    \item 这是第二项
    \item 这是第三项
\end{itemize}

\subsection{标题2}
有序号
\begin{enumerate}[ 1)]
    \item 这是第一项
    \item 这是第二项
    \item 这是第三项
\end{enumerate}

常用逻辑符号命令:
\begin{center}
\begin{tabular}{|c|c|c|}
\hline
名称      & 符号                   & 命令 \\
\hline
否定      & $\neg$                & \verb|\neg| \\
合取      & $\land$               & \verb|\land| \\
析取      & $\lor$                & \verb|\land| \\
蕴含      & $\to$                 & \verb|\to| \\
推出      & $\Rightarrow$         & \verb|\Rightarrow| \\
实质蕴涵   & $\supset$             & \verb|\supset| \\
双蕴含     & $\leftrightarrow$     & \verb|\leftrightarrow| \\
等价      & $\equiv$              & \verb|\equiv| \\
存在量词   & $\exists$             & \verb|\exists| \\
全称量词   & $\forall$             & \verb|\forall| \\
可满足,真  & $\models$            & \verb|\models| \\
语义后承   & $\vDash$,$\nvDash$   & \verb|\vDash,\nvDash| \\
句法后承   & $\vdash$,$\nvdash$   & \verb|\vdash,\nvdash| \\
属于      & $\in$,$\notin$       & \verb|\in,\notin| \\
真包含于   & $\subset$             & \verb|\subset| \\
包含于     & $\subset$            & \verb|\subset| \\
不等于     & $\neq$               & \verb|\neq| \\
小于等于   & $\leq$               & \verb|\leq| \\
大于等于   & $\geq$               & \verb|\geq| \\
下省略    & $\ldots$              & \verb|\ldots| \\
中省略    & $\cdots$              & \verb|\cdots| \\
对角省略  & $\ddots$              & \verb|\ddodts| \\
\hline 
\end{tabular}
\end{center}
更多符号对应的命令请参见:
\begin{itemize}
  \item \url{https://oeis.org/wiki/List_of_LaTeX_mathematical_symbols}或
  \item \url{https://www.ctan.org/tex-archive/info/symbols/comprehensive/}
\end{itemize}


\section{习题环境}

\begin{exs}
请证明勾股定理。
\end{exs}
\begin{proof}
这是证明。末尾后会自动添加方块以示结束。
\end{proof}

\begin{exs}
请计算 $1+2+\ldots +100$。
\end{exs}
\begin{proof}[解答]
这是解答。末尾后会自动添加方块以示结束。
\end{proof}

\end{document}